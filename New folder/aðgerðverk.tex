\documentclass[a4paper,12pt]{article}
\usepackage{multirow}
\usepackage[icelandic]{babel}
\usepackage[T1]{fontenc}
\usepackage[utf8x]{inputenc}
\usepackage{graphicx}
\usepackage{fancyhdr}
\usepackage[top=1.4in, bottom=1.4in, left=1.2in, right=1.2in]{geometry}
\usepackage[T1]{fontenc}
\usepackage{textcomp}
\usepackage{gensymb}
\usepackage{pdflscape}
\usepackage[framed,numbered,autolinebreaks,useliterate]{mcode}\lstset{language=matlab} 
\lstset{inputencoding=latin1}
\usepackage{caption}
\usepackage{here}
\usepackage{amsmath} % or simply amstext
\newcommand{\angstrom}{\textup{\AA}}
\usepackage{subcaption}
\usepackage{url} 
%\usepackage[table]{xcolor}
\usepackage[version=3]{mhchem}
\usepackage{layout}
\usepackage[parfill]{parskip}
\usepackage{arydshln}
\pagestyle{fancy}
\lhead{}
\chead{}
\rhead{IÐN401G \\ Vor 2014}
\lfoot{Háskóli Íslands}
\cfoot{}
\rfoot{\thepage}
\renewcommand{\headrulewidth}{0.0pt}
\renewcommand{\footrulewidth}{0.0pt}

\begin{document}
\begin{titlepage}
\begin{center}
\begin{minipage}{0.4\textwidth}
\begin{flushleft} 
\vspace{10mm}
        \begin{figure}[H]
      \includegraphics[width=0.4\linewidth]{HI_merki.jpg}
        \label{fig: logo}
        \end{figure}
\end{flushleft}
\end{minipage}
\begin{minipage}{0.4\textwidth}
\begin{flushright} \large
\textsc{Háskóli Íslands\\Verkfræðideild}
\end{flushright}
\end{minipage}

\vspace{4cm}



%\includegraphics[width=0.15\textwidth]{HI_merki.jpg}\\[1cm]


{\textsc{\Large Aðgerðagreining}\\[0.5cm]}

\vspace{0.8cm}
% Title
\begin{center}
\rule{1\textwidth}{3pt}

{\textsc{ \LARGE \bfseries Bestun stundatöflu í stokkakerfi}} \\[0.4cm]



\rule{1\textwidth}{4pt}
\vspace{2cm}

\end{center}
\today

\vfill

\begin{minipage}{0.4\textwidth}
\begin{flushleft} 
\vspace{1cm}
\emph{Kennari:}\\
\textsc{Tómas Philip Rúnarsson}\\
\end{flushleft}
\end{minipage}
\begin{minipage}{0.4\textwidth}
\begin{flushright} 
\emph{Nemendur:}\\
Baldur Geir Gunnarsson\\
Einar Halldórsson\\
Gestur Hvannberg\\
Oddur Vilhjálmsson\\
Trausti Kouichi Ásgeirsson
\end{flushright}
\end{minipage}
% Bottom of the page
%{\large Vorönn, 2014}

\end{center}

\newpage
\thispagestyle{empty} \mbox{}
\vfill
%\begin{center}\textit{\thesisdedication}\end{center} \vspace*{5cm}
\vfill 
\thispagestyle{empty}
\cleardoublepage



\end{titlepage}

\title{Bestun stundatöflu í stokkakerfi}
\author{Baldur Geir Gunnarsson, Einar Halldórsson, Gestur Hvannberg,\\ Oddur Vilhjálmsson, Trausti Kouichi Ásgeirsson}
\maketitle

\section{Ágrip}
Verkfræði og náttúruvísindasvið Háskóla Íslands notast við stokkakerfi við stundatöflugerð. Samtals eru 7 stokkar á hverri önn og raða þarf áföngum niður á þá. Markmið okkar var að hanna stundatöflur fyrir allar annir í Eðlisfræði og kanna eiginlega þeirrar lausnar. Stokkarnir líta svona út í dag:
\begin{center}
\includegraphics{stokkur}
\end{center}

\section{Inngangur/bakgrunnur}



\section{Niðurstöður, niðurlag og tillögur}


\section{Aðferðir}

\section{Almenn umfjöllun....sleppa??}


\section{Heimildir}



\pagebreak
\section{Viðauki}



\end{document}


